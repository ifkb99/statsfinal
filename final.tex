\PassOptionsToPackage{unicode=true}{hyperref} % options for packages loaded elsewhere
\PassOptionsToPackage{hyphens}{url}
%
\documentclass[]{article}
\usepackage{lmodern}
\usepackage{amssymb,amsmath}
\usepackage{ifxetex,ifluatex}
\usepackage{fixltx2e} % provides \textsubscript
\ifnum 0\ifxetex 1\fi\ifluatex 1\fi=0 % if pdftex
  \usepackage[T1]{fontenc}
  \usepackage[utf8]{inputenc}
  \usepackage{textcomp} % provides euro and other symbols
\else % if luatex or xelatex
  \usepackage{unicode-math}
  \defaultfontfeatures{Ligatures=TeX,Scale=MatchLowercase}
\fi
% use upquote if available, for straight quotes in verbatim environments
\IfFileExists{upquote.sty}{\usepackage{upquote}}{}
% use microtype if available
\IfFileExists{microtype.sty}{%
\usepackage[]{microtype}
\UseMicrotypeSet[protrusion]{basicmath} % disable protrusion for tt fonts
}{}
\IfFileExists{parskip.sty}{%
\usepackage{parskip}
}{% else
\setlength{\parindent}{0pt}
\setlength{\parskip}{6pt plus 2pt minus 1pt}
}
\usepackage{hyperref}
\hypersetup{
            pdftitle={Final Project},
            pdfauthor={Ian Baker, Loughlin Claus, Zack Schieberl},
            pdfborder={0 0 0},
            breaklinks=true}
\urlstyle{same}  % don't use monospace font for urls
\usepackage[margin=2cm]{geometry}
\usepackage{color}
\usepackage{fancyvrb}
\newcommand{\VerbBar}{|}
\newcommand{\VERB}{\Verb[commandchars=\\\{\}]}
\DefineVerbatimEnvironment{Highlighting}{Verbatim}{commandchars=\\\{\}}
% Add ',fontsize=\small' for more characters per line
\usepackage{framed}
\definecolor{shadecolor}{RGB}{248,248,248}
\newenvironment{Shaded}{\begin{snugshade}}{\end{snugshade}}
\newcommand{\AlertTok}[1]{\textcolor[rgb]{0.94,0.16,0.16}{#1}}
\newcommand{\AnnotationTok}[1]{\textcolor[rgb]{0.56,0.35,0.01}{\textbf{\textit{#1}}}}
\newcommand{\AttributeTok}[1]{\textcolor[rgb]{0.77,0.63,0.00}{#1}}
\newcommand{\BaseNTok}[1]{\textcolor[rgb]{0.00,0.00,0.81}{#1}}
\newcommand{\BuiltInTok}[1]{#1}
\newcommand{\CharTok}[1]{\textcolor[rgb]{0.31,0.60,0.02}{#1}}
\newcommand{\CommentTok}[1]{\textcolor[rgb]{0.56,0.35,0.01}{\textit{#1}}}
\newcommand{\CommentVarTok}[1]{\textcolor[rgb]{0.56,0.35,0.01}{\textbf{\textit{#1}}}}
\newcommand{\ConstantTok}[1]{\textcolor[rgb]{0.00,0.00,0.00}{#1}}
\newcommand{\ControlFlowTok}[1]{\textcolor[rgb]{0.13,0.29,0.53}{\textbf{#1}}}
\newcommand{\DataTypeTok}[1]{\textcolor[rgb]{0.13,0.29,0.53}{#1}}
\newcommand{\DecValTok}[1]{\textcolor[rgb]{0.00,0.00,0.81}{#1}}
\newcommand{\DocumentationTok}[1]{\textcolor[rgb]{0.56,0.35,0.01}{\textbf{\textit{#1}}}}
\newcommand{\ErrorTok}[1]{\textcolor[rgb]{0.64,0.00,0.00}{\textbf{#1}}}
\newcommand{\ExtensionTok}[1]{#1}
\newcommand{\FloatTok}[1]{\textcolor[rgb]{0.00,0.00,0.81}{#1}}
\newcommand{\FunctionTok}[1]{\textcolor[rgb]{0.00,0.00,0.00}{#1}}
\newcommand{\ImportTok}[1]{#1}
\newcommand{\InformationTok}[1]{\textcolor[rgb]{0.56,0.35,0.01}{\textbf{\textit{#1}}}}
\newcommand{\KeywordTok}[1]{\textcolor[rgb]{0.13,0.29,0.53}{\textbf{#1}}}
\newcommand{\NormalTok}[1]{#1}
\newcommand{\OperatorTok}[1]{\textcolor[rgb]{0.81,0.36,0.00}{\textbf{#1}}}
\newcommand{\OtherTok}[1]{\textcolor[rgb]{0.56,0.35,0.01}{#1}}
\newcommand{\PreprocessorTok}[1]{\textcolor[rgb]{0.56,0.35,0.01}{\textit{#1}}}
\newcommand{\RegionMarkerTok}[1]{#1}
\newcommand{\SpecialCharTok}[1]{\textcolor[rgb]{0.00,0.00,0.00}{#1}}
\newcommand{\SpecialStringTok}[1]{\textcolor[rgb]{0.31,0.60,0.02}{#1}}
\newcommand{\StringTok}[1]{\textcolor[rgb]{0.31,0.60,0.02}{#1}}
\newcommand{\VariableTok}[1]{\textcolor[rgb]{0.00,0.00,0.00}{#1}}
\newcommand{\VerbatimStringTok}[1]{\textcolor[rgb]{0.31,0.60,0.02}{#1}}
\newcommand{\WarningTok}[1]{\textcolor[rgb]{0.56,0.35,0.01}{\textbf{\textit{#1}}}}
\usepackage{graphicx,grffile}
\makeatletter
\def\maxwidth{\ifdim\Gin@nat@width>\linewidth\linewidth\else\Gin@nat@width\fi}
\def\maxheight{\ifdim\Gin@nat@height>\textheight\textheight\else\Gin@nat@height\fi}
\makeatother
% Scale images if necessary, so that they will not overflow the page
% margins by default, and it is still possible to overwrite the defaults
% using explicit options in \includegraphics[width, height, ...]{}
\setkeys{Gin}{width=\maxwidth,height=\maxheight,keepaspectratio}
\setlength{\emergencystretch}{3em}  % prevent overfull lines
\providecommand{\tightlist}{%
  \setlength{\itemsep}{0pt}\setlength{\parskip}{0pt}}
\setcounter{secnumdepth}{0}
% Redefines (sub)paragraphs to behave more like sections
\ifx\paragraph\undefined\else
\let\oldparagraph\paragraph
\renewcommand{\paragraph}[1]{\oldparagraph{#1}\mbox{}}
\fi
\ifx\subparagraph\undefined\else
\let\oldsubparagraph\subparagraph
\renewcommand{\subparagraph}[1]{\oldsubparagraph{#1}\mbox{}}
\fi

% set default figure placement to htbp
\makeatletter
\def\fps@figure{htbp}
\makeatother

\usepackage{etoolbox}
\makeatletter
\providecommand{\subtitle}[1]{% add subtitle to \maketitle
  \apptocmd{\@title}{\par {\large #1 \par}}{}{}
}
\makeatother
% https://github.com/rstudio/rmarkdown/issues/337
\let\rmarkdownfootnote\footnote%
\def\footnote{\protect\rmarkdownfootnote}

% https://github.com/rstudio/rmarkdown/pull/252
\usepackage{titling}
\setlength{\droptitle}{-2em}

\pretitle{\vspace{\droptitle}\centering\huge}
\posttitle{\par}

\preauthor{\centering\large\emph}
\postauthor{\par}

\predate{\centering\large\emph}
\postdate{\par}

\title{Final Project}
\author{Ian Baker, Loughlin Claus, Zack Schieberl}
\date{12/7/2019}

\begin{document}
\maketitle

\hypertarget{pledge}{%
\subsection{Pledge}\label{pledge}}

I pledge my honor that I have abided by the Stevens Honor System - Ian
Baker, Loughlin Claus, Zack Schieberl

\hypertarget{section}{%
\subsection{11.53}\label{section}}

\begin{Shaded}
\begin{Highlighting}[]
\NormalTok{cheese <-}\StringTok{ }\KeywordTok{as.matrix}\NormalTok{(}\KeywordTok{read.csv2}\NormalTok{(}\StringTok{"cheese.csv"}\NormalTok{, }\DataTypeTok{header =} \OtherTok{TRUE}\NormalTok{, }\DataTypeTok{sep =} \StringTok{","}\NormalTok{))}

\NormalTok{cheeseCols <-}\StringTok{ }\KeywordTok{colnames}\NormalTok{(cheese)}
\ControlFlowTok{for}\NormalTok{ (col }\ControlFlowTok{in}\NormalTok{ cheeseCols) \{}
\NormalTok{  cur <-}\StringTok{ }\KeywordTok{as.numeric}\NormalTok{(cheese[, col])}
  \CommentTok{# mean, median, sd, iqr}
\NormalTok{  out <-}\StringTok{ }\KeywordTok{c}\NormalTok{(}\KeywordTok{paste}\NormalTok{(}\StringTok{"Type:"}\NormalTok{, col), }\KeywordTok{paste}\NormalTok{(}\StringTok{"Mean:"}\NormalTok{, }\KeywordTok{round}\NormalTok{(}\KeywordTok{mean}\NormalTok{(cur), }\DecValTok{2}\NormalTok{)),}
           \KeywordTok{paste}\NormalTok{(}\StringTok{"Median:"}\NormalTok{, }\KeywordTok{round}\NormalTok{(}\KeywordTok{median}\NormalTok{(cur), }\DecValTok{2}\NormalTok{)), }\KeywordTok{paste}\NormalTok{(}\StringTok{"SD:"}\NormalTok{, }\KeywordTok{round}\NormalTok{(}\KeywordTok{sd}\NormalTok{(cur), }\DecValTok{2}\NormalTok{)),}
           \KeywordTok{paste}\NormalTok{(}\StringTok{"IQR:"}\NormalTok{, }\KeywordTok{round}\NormalTok{(}\KeywordTok{IQR}\NormalTok{(cur), }\DecValTok{2}\NormalTok{)))}
  \KeywordTok{print}\NormalTok{(}\KeywordTok{format}\NormalTok{(out, }\DataTypeTok{justify =} \StringTok{"left"}\NormalTok{, }\DataTypeTok{trim =} \OtherTok{TRUE}\NormalTok{))}
  \CommentTok{# stemplot}
  \KeywordTok{stem}\NormalTok{(cur)}
  \CommentTok{# normal quantile plot}
  \KeywordTok{qqnorm}\NormalTok{(cur, }\DataTypeTok{main =}\NormalTok{ col)}
  \KeywordTok{qqline}\NormalTok{(cur)}
\NormalTok{\}}
\end{Highlighting}
\end{Shaded}

\begin{verbatim}
## [1] "Type: taste  " "Mean: 24.53  " "Median: 20.95" "SD: 16.26    "
## [5] "IQR: 23.15   "
## 
##   The decimal point is 1 digit(s) to the right of the |
## 
##   0 | 11666
##   1 | 223456788
##   2 | 112667
##   3 | 25799
##   4 | 18
##   5 | 577
\end{verbatim}

\begin{verbatim}
## [1] "Type: acetic" "Mean: 5.5   " "Median: 5.42" "SD: 0.57    " "IQR: 0.65   "
## 
##   The decimal point is 1 digit(s) to the left of the |
## 
##   44 | 846
##   46 | 69
##   48 | 0
##   50 | 6
##   52 | 4450377
##   54 | 146
##   56 | 046
##   58 | 069
##   60 | 4858
##   62 | 7
##   64 | 56
\end{verbatim}

\begin{verbatim}
## [1] "Type: h2s   " "Mean: 5.94  " "Median: 5.33" "SD: 2.13    " "IQR: 3.6    "
## 
##   The decimal point is at the |
## 
##    2 | 
##    3 | 01278999
##    4 | 27899
##    5 | 024
##    6 | 1278
##    7 | 0569
##    8 | 07
##    9 | 126
##   10 | 2
\end{verbatim}

\begin{verbatim}
## [1] "Type: lactic" "Mean: 1.44  " "Median: 1.45" "SD: 0.3     " "IQR: 0.42   "
## 
##   The decimal point is 1 digit(s) to the left of the |
## 
##    8 | 69
##   10 | 68956
##   12 | 5599013
##   14 | 4692378
##   16 | 38248
##   18 | 109
##   20 | 1
\end{verbatim}

\includegraphics{final_files/figure-latex/unnamed-chunk-1-1.pdf}
\includegraphics{final_files/figure-latex/unnamed-chunk-1-2.pdf}
\includegraphics{final_files/figure-latex/unnamed-chunk-1-3.pdf}
\includegraphics{final_files/figure-latex/unnamed-chunk-1-4.pdf}

While H2S and Taste have some right skew, and Acetic has two peaks, the
data all appears to be relatively normal. There are no outliers in the
data.

\hypertarget{section-1}{%
\subsection{11.54}\label{section-1}}

\begin{Shaded}
\begin{Highlighting}[]
\ControlFlowTok{for}\NormalTok{ (col }\ControlFlowTok{in}\NormalTok{ cheeseCols) \{}
\NormalTok{  colIdx <-}\StringTok{ }\KeywordTok{grep}\NormalTok{(col, cheeseCols)}
\NormalTok{  col1Data <-}\StringTok{ }\KeywordTok{as.numeric}\NormalTok{(cheese[, col])}
  \ControlFlowTok{for}\NormalTok{ (col2 }\ControlFlowTok{in}\NormalTok{ cheeseCols) \{}
    \ControlFlowTok{if}\NormalTok{ (colIdx }\OperatorTok{<}\StringTok{ }\KeywordTok{grep}\NormalTok{(col2, cheeseCols)) \{}
\NormalTok{      col2Data <-}\StringTok{ }\KeywordTok{as.numeric}\NormalTok{(cheese[, col2])}
      \KeywordTok{plot}\NormalTok{(col1Data, col2Data, }\DataTypeTok{xlab =}\NormalTok{ col, }\DataTypeTok{ylab =}\NormalTok{ col2)}
\NormalTok{      correl <-}\StringTok{ }\KeywordTok{cor.test}\NormalTok{(col1Data, col2Data)}
      \KeywordTok{cat}\NormalTok{(}\StringTok{"Correlation between"}\NormalTok{, col, }\StringTok{"and"}\NormalTok{, col2, }\StringTok{"is:"}\NormalTok{, correl}\OperatorTok{$}\NormalTok{estimate,}
          \StringTok{"with a p-value of"}\NormalTok{, correl}\OperatorTok{$}\NormalTok{p.value, }\StringTok{"}\CharTok{\textbackslash{}n}\StringTok{"}\NormalTok{)}
\NormalTok{    \}}
\NormalTok{  \}}
\NormalTok{\}}
\end{Highlighting}
\end{Shaded}

\begin{verbatim}
## Correlation between taste and acetic is: 0.5495393 with a p-value of 0.001658192
\end{verbatim}

\begin{verbatim}
## Correlation between taste and h2s is: 0.7557523 with a p-value of 1.373783e-06
\end{verbatim}

\begin{verbatim}
## Correlation between taste and lactic is: 0.7042362 with a p-value of 1.405117e-05
\end{verbatim}

\begin{verbatim}
## Correlation between acetic and h2s is: 0.6179559 with a p-value of 0.0002739173
\end{verbatim}

\begin{verbatim}
## Correlation between acetic and lactic is: 0.6037826 with a p-value of 0.0004113657
\end{verbatim}

\begin{verbatim}
## Correlation between h2s and lactic is: 0.6448123 with a p-value of 0.0001198401
\end{verbatim}

\includegraphics{final_files/figure-latex/unnamed-chunk-2-1.pdf}
\includegraphics{final_files/figure-latex/unnamed-chunk-2-2.pdf}
\includegraphics{final_files/figure-latex/unnamed-chunk-2-3.pdf}
\includegraphics{final_files/figure-latex/unnamed-chunk-2-4.pdf}
\includegraphics{final_files/figure-latex/unnamed-chunk-2-5.pdf}
\includegraphics{final_files/figure-latex/unnamed-chunk-2-6.pdf}

\hypertarget{section-2}{%
\subsection{11.55}\label{section-2}}

\begin{Shaded}
\begin{Highlighting}[]
\NormalTok{tasteCol <-}\StringTok{ }\KeywordTok{as.numeric}\NormalTok{(cheese[, }\StringTok{"taste"}\NormalTok{])}
\NormalTok{aceticCol <-}\StringTok{ }\KeywordTok{as.numeric}\NormalTok{(cheese[, }\StringTok{"acetic"}\NormalTok{])}
\NormalTok{tasteVsAcetic <-}\StringTok{ }\KeywordTok{lm}\NormalTok{(tasteCol }\OperatorTok{~}\StringTok{ }\NormalTok{aceticCol, }\KeywordTok{data.frame}\NormalTok{(cheese))}
\KeywordTok{plot}\NormalTok{(aceticCol, tasteCol, }\DataTypeTok{xlab =} \StringTok{"Acetic"}\NormalTok{, }\DataTypeTok{ylab =} \StringTok{"Taste"}\NormalTok{)}
\KeywordTok{abline}\NormalTok{(tasteVsAcetic)}
\end{Highlighting}
\end{Shaded}

\includegraphics{final_files/figure-latex/unnamed-chunk-3-1.pdf}

\begin{Shaded}
\begin{Highlighting}[]
\NormalTok{tVsAResiduals <-}\StringTok{ }\KeywordTok{residuals}\NormalTok{(tasteVsAcetic)}
\KeywordTok{plot}\NormalTok{(cheese[, }\StringTok{"h2s"}\NormalTok{], tVsAResiduals, }\DataTypeTok{xlab =} \StringTok{"H2S"}\NormalTok{, }\DataTypeTok{ylab =} \StringTok{"TvA Residuals"}\NormalTok{)}
\KeywordTok{plot}\NormalTok{(cheese[, }\StringTok{"lactic"}\NormalTok{], tVsAResiduals, }\DataTypeTok{xlab =} \StringTok{"Lactic"}\NormalTok{, }\DataTypeTok{ylab =} \StringTok{"TvA Residuals"}\NormalTok{)}
\end{Highlighting}
\end{Shaded}

\includegraphics{final_files/figure-latex/unnamed-chunk-4-1.pdf}
\includegraphics{final_files/figure-latex/unnamed-chunk-4-2.pdf}

The residuals both have a normal distribution and seem to be positively
associated with Lactic and H2S.

\hypertarget{section-3}{%
\subsection{11.56}\label{section-3}}

\begin{Shaded}
\begin{Highlighting}[]
\NormalTok{h2sCol <-}\StringTok{ }\KeywordTok{as.numeric}\NormalTok{(cheese[, }\StringTok{"h2s"}\NormalTok{])}
\NormalTok{tasteVsH2S <-}\StringTok{ }\KeywordTok{lm}\NormalTok{(tasteCol }\OperatorTok{~}\StringTok{ }\NormalTok{h2sCol, }\KeywordTok{data.frame}\NormalTok{(cheese))}
\KeywordTok{plot}\NormalTok{(h2sCol, tasteCol, }\DataTypeTok{xlab =} \StringTok{"H2S"}\NormalTok{, }\DataTypeTok{ylab =} \StringTok{"Taste"}\NormalTok{)}
\KeywordTok{abline}\NormalTok{(tasteVsH2S)}
\end{Highlighting}
\end{Shaded}

\includegraphics{final_files/figure-latex/unnamed-chunk-5-1.pdf}

\begin{Shaded}
\begin{Highlighting}[]
\NormalTok{tVsHResiduals <-}\StringTok{ }\KeywordTok{residuals}\NormalTok{(tasteVsH2S)}
\KeywordTok{plot}\NormalTok{(cheese[, }\StringTok{"acetic"}\NormalTok{], tVsHResiduals, }\DataTypeTok{xlab =} \StringTok{"Acetic"}\NormalTok{, }\DataTypeTok{ylab =} \StringTok{"TvH2S Residuals"}\NormalTok{)}
\KeywordTok{plot}\NormalTok{(cheese[, }\StringTok{"lactic"}\NormalTok{], tVsHResiduals, }\DataTypeTok{xlab =} \StringTok{"Lactic"}\NormalTok{, }\DataTypeTok{ylab =} \StringTok{"TvH2S Residuals"}\NormalTok{)}
\end{Highlighting}
\end{Shaded}

\includegraphics{final_files/figure-latex/unnamed-chunk-6-1.pdf}
\includegraphics{final_files/figure-latex/unnamed-chunk-6-2.pdf}

From the graphs there appears to be no correlation between the residuals
and other variables.

\hypertarget{section-4}{%
\subsection{11.57}\label{section-4}}

\begin{Shaded}
\begin{Highlighting}[]
\NormalTok{lacticCol <-}\StringTok{ }\KeywordTok{as.numeric}\NormalTok{(cheese[, }\StringTok{"lactic"}\NormalTok{])}
\NormalTok{tasteVsLactic <-}\StringTok{ }\KeywordTok{lm}\NormalTok{(tasteCol }\OperatorTok{~}\StringTok{ }\NormalTok{lacticCol, }\KeywordTok{data.frame}\NormalTok{(cheese))}
\KeywordTok{plot}\NormalTok{(lacticCol, tasteCol, }\DataTypeTok{xlab =} \StringTok{"Lactic"}\NormalTok{, }\DataTypeTok{ylab =} \StringTok{"Taste"}\NormalTok{)}
\KeywordTok{abline}\NormalTok{(tasteVsLactic)}
\end{Highlighting}
\end{Shaded}

\includegraphics{final_files/figure-latex/unnamed-chunk-7-1.pdf}

\begin{Shaded}
\begin{Highlighting}[]
\NormalTok{tVsLResiduals <-}\StringTok{ }\KeywordTok{residuals}\NormalTok{(tasteVsLactic)}
\KeywordTok{plot}\NormalTok{(cheese[, }\StringTok{"acetic"}\NormalTok{], tVsLResiduals, }\DataTypeTok{xlab =} \StringTok{"Acetic"}\NormalTok{, }\DataTypeTok{ylab =} \StringTok{"TvL Residuals"}\NormalTok{)}
\KeywordTok{plot}\NormalTok{(cheese[, }\StringTok{"h2s"}\NormalTok{], tVsLResiduals, }\DataTypeTok{xlab =} \StringTok{"H2S"}\NormalTok{, }\DataTypeTok{ylab =} \StringTok{"TvL Residuals"}\NormalTok{)}
\end{Highlighting}
\end{Shaded}

\includegraphics{final_files/figure-latex/unnamed-chunk-8-1.pdf}
\includegraphics{final_files/figure-latex/unnamed-chunk-8-2.pdf}

Again, there appears to be no correlation between the residuals and the
other variables.

\hypertarget{section-5}{%
\subsection{11.58}\label{section-5}}

\begin{Shaded}
\begin{Highlighting}[]
\CommentTok{# table of F stat, P-val, R^2, and standard dev estimate}
\NormalTok{tVsASum <-}\StringTok{ }\KeywordTok{summary}\NormalTok{(tasteVsAcetic)}
\NormalTok{tVsHSum <-}\StringTok{ }\KeywordTok{summary}\NormalTok{(tasteVsH2S)}
\NormalTok{tVsLSum <-}\StringTok{ }\KeywordTok{summary}\NormalTok{(tasteVsLactic)}


\CommentTok{# regression equations}
\end{Highlighting}
\end{Shaded}

The intercepts in the three equations are different because the
explanitory variables all have different values, leading to the points
being plotted in different places. Since the linear equations are
estimating the best fit line for these datapoints, it is only natural
that the differing data produces different intercepts.

\hypertarget{section-6}{%
\subsection{11.59}\label{section-6}}

TODO

\hypertarget{section-7}{%
\subsection{11.60}\label{section-7}}

TODO

\hypertarget{section-8}{%
\subsection{11.61}\label{section-8}}

TODO

\end{document}
